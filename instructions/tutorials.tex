
\begin{frame}{example}
	\tableofcontents
\end{frame}

\section{开始}

	\subsection{几种版本}

		\begin{frame}[fragile]
			\frametitle{几种版本}
			在$\id{main.tex}$的前四行提供以下\LaTeX 类

\begin{lstlisting}[language=tex]
% \documentclass[ebook]{myslide} % 移动设备
\documentclass[slide]{myslide} % 幻灯片展示
% \documentclass[slide,handout]{myslide} % 幻灯片无动画
% \documentclass[plain]{myslide} % A4打印
\end{lstlisting}
分别取消注释可以的到对应类的模板.

对于linux可以直接通过
\begin{lstlisting}[language=bash]
make all
\end{lstlisting}
运行脚本自动创建所有版本

		\end{frame}

	\subsection{说明}

		\begin{frame}{一些说明}
			\begin{itemize}
				\item article模式忽略frame标题
			\end{itemize}
		\end{frame}
		\mode<article>{
			由于使用ignorenonframetext会使得standout出现问题,
			故只在article中显示的要用
			\lstinline{\\mode<article>{}}
			避免slide出现问题。

			例如,这段文字在幻灯片模式中看不见
		}

\section{其他新命令}
	\subsection{alertshow}

		\begin{frame}[fragile]
			\frametitle{alertShow}
			\begin{lstlisting}[language=tex]
\alertShow{2}{这些文字会在slide第二张出现并强调,其他版本不强调}
\end{lstlisting}
\alertShow{2}{这些文字会在slide第二张出现并强调,其他版本不强调}
		\end{frame}

	\subsection{onlyFocus}
		\begin{frame}[fragile]
			\frametitle{onlyFocus}
			\begin{lstlisting}[language=tex]
\onlyFocus{2}{这些文字只会在slide第二张出现并强调,
未出现不占空间,其他版本不显示}
\end{lstlisting}
\onlyFocus{2}{这些文字只会在slide第二张出现并强调,未出现不占空间,其他版本不显示}
		\end{frame}
	
	\subsection{cite}
		\begin{frame}[fragile]{cite}
			头文件添加添加,编译使用biber
\begin{lstlisting}[language=tex]
\usepackage[backend=biber,
	style=alphabetic]{biblatex}
\end{lstlisting}
尝试引用\cite{github}
		\end{frame}
	\subsection{bibFrame}
		\begin{frame}[fragile]{bibFrame}
			产生一页文献摘录
\begin{lstlisting}[language=tex]
\bibFrame{github}{some section}{text from the bibliography}
\end{lstlisting}
		\end{frame}

		\bibFrame{github}{some section}{
			text from the bibliography
		}
	\subsection{中英}
		\begin{frame}[fragile]{中英}
			\begin{lstlisting}[language=tex]
\zhEn{中文}{English}%
\end{lstlisting}
\zhEn{中文}{English}%
\zhEn{中文}{English}%

		\end{frame}

	\subsection{formal}

		\begin{frame}
			\begin{formal}
				formal
			\end{formal}
		\end{frame}

	\subsection{隐藏logo}
		\hideLogo
		\begin{frame}
			frame前使用命令
			\lstinline{\hideLogo}隐藏logo

			frame后使用命令
			\lstinline{\showLogo}
			恢复logo
		\end{frame}
		\showLogo
\section{代码强调}

	\begin{frame}[fragile]
		\frametitle{linehighlight}
		\begin{linehighlight}{
				\only<1,3>{ % Only on slides 1 and 3 (beamer stuff)
							\highline{1,3,5,12} % highlight code lines 1,3 and 5,12.
				}
				\only<2>{ % Only on slide 2 (beamer)
							\highline{2,...,12} % highlight lines 2 to 12.
				}
			}
			% put your \lstinputlisting{} or \begin{lstlisting}...\end{lstlisting} here
			\lstinputlisting[language=tex,style=highlight]{src/highline.tex}
		\end{linehighlight}

	\end{frame}

\begin{frame}[standout]
	测试standout
\end{frame}

\section{test}
	\begin{frame}{test theorem}
		\begin{theorem}
			a test theorem
			\label{th:test-th}
		\end{theorem}
		\cref{th:test-th}
	\end{frame}

	\begin{frame}{equation}
		\begin{equation}
			E = mc^2
			% \label{eq:}
		\end{equation}
	\end{frame}

\section{项目地址}
\hideLogo
\begin{frame}{项目地址}
	\url{https://github.com/chenboshuo/myslide_template}
\end{frame}
