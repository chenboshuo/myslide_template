% \documentclass[ebook]{myslide} % 移动设备
\documentclass[slide]{myslide} % 幻灯片展示
% \documentclass[slide,handout]{myslide} % 幻灯片无动画
% \documentclass[plain]{myslide} % A4打印
%% 以上内容更改后需要修改 make.sh

\title{关于模板的使用说明}
\date{\today}
\author{陈伯硕}
\institute{github}
\logo{\includegraphics[scale=0.7]{logo}}


\begin{document}
\maketitle

\begin{frame}{example}
	\tableofcontents
\end{frame}

\section{开始}

\subsection{几种版本}



\begin{frame}[fragile]
  \frametitle{几种版本}
		在$\id{main.tex}$的前四行提供以下\LaTeX 类

\begin{lstlisting}[language=python]
% \documentclass[ebook]{myslide} % 移动设备
\documentclass[slide]{myslide} % 幻灯片展示
% \documentclass[slide,handout]{myslide} % 幻灯片无动画
% \documentclass[plain]{myslide} % A4打印
\end{lstlisting}
	分别取消注释可以的到对应类的模板.

	对于linux可以直接通过
\begin{lstlisting}[language=bash]
sh make.sh
\end{lstlisting}
	运行脚本自动创建所有版本

\end{frame}

\subsubsection{说明}

	\begin{frame}{一些说明}
		\begin{itemize}
			\item beamer模式忽略frame之外的文字
			\item article模式忽略frame标题
		\end{itemize}

	\end{frame}
	例如这段文字在幻灯片模式中看不见
	忽略幻灯片之外的文字

\subsection{其他新命令}




\end{document}
