% \documentclass[ebook]{myslide} % 移动设备
\documentclass[slide]{myslide} % 幻灯片展示
% \documentclass[slide,handout]{myslide} % 幻灯片无动画
% \documentclass[plain]{myslide} % A4打印
%% 以上内容更改后需要修改 make.sh

\title{关于模板的使用说明}
\date{\today}
\author{陈伯硕}
\institute{github}
\logo{\includegraphics[scale=0.7]{logo}}

% references
\usepackage[alldates=iso]{biblatex}
\bibliography{reference}

\begin{document}
\begin{frame}
	\maketitle
\end{frame}

\begin{frame}{example}
	\tableofcontents
\end{frame}

\section{开始}

\subsection{几种版本}

\begin{frame}[fragile]
  \frametitle{几种版本}
		在$\id{main.tex}$的前四行提供以下\LaTeX 类

\begin{lstlisting}[language=tex]
% \documentclass[ebook]{myslide} % 移动设备
\documentclass[slide]{myslide} % 幻灯片展示
% \documentclass[slide,handout]{myslide} % 幻灯片无动画
% \documentclass[plain]{myslide} % A4打印
\end{lstlisting}
	分别取消注释可以的到对应类的模板.

	对于linux可以直接通过
\begin{lstlisting}[language=bash]
make all
\end{lstlisting}
	运行脚本自动创建所有版本

\end{frame}

\subsubsection{说明}

	\begin{frame}{一些说明}
		\begin{itemize}
			\item beamer模式忽略frame之外的文字
			\item article模式忽略frame标题
		\end{itemize}

	\end{frame}
	\mode<article>{
		由于使用ignorenonframetext会使得standout出现问题,
		故劲仔article中显示的要用
		\lstinline{\\mode<article>{}}
		避免slide出现问题。

		例如,这段文字在幻灯片模式中看不见
	}

\section{其他新命令}
	\subsubsection{alertshow}

\begin{frame}[fragile]
	\frametitle{alertshow}
\begin{lstlisting}[language=tex]
\alertshow{2}{这些文字会在slide第二张出现并强调,其他版本不强调}
\end{lstlisting}
	\alertshow{2}{这些文字会在slide第二张出现并强调,其他版本不强调}
\end{frame}

\subsection{onlyfocus}
\begin{frame}[fragile]
	\frametitle{onlyfocus}
\begin{lstlisting}[language=tex]
\onlyfocus{2}{这些文字只会在slide第二张出现并强调,
未出现不占空间,其他版本不显示}
\end{lstlisting}
	\onlyfocus{2}{这些文字只会在slide第二张出现并强调,未出现不占空间,其他版本不显示}
\end{frame}

\begin{frame}
	尝试引用\cite{github}
\end{frame}

\subsection{formal}

\begin{frame}
	\begin{formal}
		formal
	\end{formal}
\end{frame}

\section{代码强调}

\begin{frame}[fragile]
	\frametitle{linehighlight}
	\begin{linehighlight}{
			\only<1,3>{ % Only on slides 1 and 3 (beamer stuff)
						\highline{1,3,5,12} % highlight code lines 1,3 and 5,12.
			}
			\only<2>{ % Only on slide 2 (beamer)
						\highline{2,...,12} % highlight lines 2 to 12.
			}
		}
		% put your \lstinputlisting{} or \begin{lstlisting}...\end{lstlisting} here
		\lstinputlisting[language=tex,style=highlight]{src/highline.tex}
	\end{linehighlight}

\end{frame}

\begin{frame}[standout]
	测试standout
\end{frame}

\begin{frame}{项目地址}
	\url{https://github.com/chenboshuo/myslide_template}
\end{frame}

\addcontentsline{toc}{section}{References}
	\begin{frame}[allowframebreaks]
	\frametitle{References}
	% \bibsetup{style=alphabetic} % 设置文献引用风格
		\nocite{*}
		\printbibliography
	\end{frame}

\end{document}
